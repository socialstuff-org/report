%!TEX root = ../report.tex
\chapter{Conclusion}\label{ch:conclusion}
In the \ac{sofa} module of the seventh semester, we managed to create a solid foundation for future development of
Trale.
The Trale team had a vision of a completely encrypted and decentralized chat application, that would not allow any
third party to access private chats.

Currently, large service providers of social media applications, like WhatsApp, have to grant access to governments upon
request.
Trale will not even be capable of doing so, as the whole concept foundation lies on privacy, security and anonymity by
design.
In order to make the application popular, it was important to create a user-friendly \ac{ui}, as well as an easy to set
up application, where the user will not even take note of the complex encryption algorithms taking place in the
background.
We developed this \ac{ui} for both, the end-user and server administrator.
Users are actually able to send end-to-end encrypted messages, that can only be read by the intended sender and receiver
and neither be read by third parties, nor by Trale itself.

We achieved this by conducting excessive research on encryption algorithms and developing our own handshake for key
exchanges.
Each member of the Trale team managed to enhance their competences in different areas.
Dave Hoevenaars and Maurits van der Zee used their newly achieved project management knowledge to keep the project in
a steady direction and to keep the original vision, whilst also mainly pushing development on the end-user's frontend.
In addition, Maurits van der Zee was involved in the research aspect of finding and elaborating on proper encryption
algorithms in the design artefacts together with J\"orn Neumeyer.
Furthermore, J\"orn Neumeyer was deeply involved in the network architecture as well as product owner / innovation
leader.
In the process, he became an expert for encryption technology.
J\"orn Neumeyer and Tobias Jansen worked on the backend services implementation in parallel.
Tobias Jansen and Malte Castner were in charge of the admin panel.
Malte Castner designed the frontend of the admin panel and improved his Electron and Angular skills, while Tobias
Jansen was in charge of the required backend services and improved his TypeScript skills during this process.

Now, that the project is over, and all the aforementioned features, required for an \ac{mvp} are completely
implemented.
The resulting application of the \ac{sofa} project can now, thanks to the excessive documentation of the project,
function as a solid basis to be extended upon by future \ac{sofa} groups.

This brings huge potential for both, the upcoming \ac{sofa} groups, to expand their knowledge, as well as the end-user,
to finally have a secure messaging service everyone desires.
