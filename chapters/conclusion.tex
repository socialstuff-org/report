%!TEX root = ../report.tex
\chapter{Conclusion}\label{ch:conclusion}
In the Software factory of the seventh semester we managed to create a solid foundation for future development of Trale.
The Trale team had a vision of a completely encrypted decentralized chat application, that would not allow any third
party to access private chats.

Currently big chat applications like WhatsApp have to hand allow access to governments.
Trale will not allow this, Trale will not even be capable of doing so even if they wanted to.
In order to make the application popular it was important to create a user friendly, easy to set up application, where
the user wont even take note of the complex encryption algorithm taking place in the background.
We developed this user interface for both, the normal user and the administrator.
Users are actually able to send end-to-end encrypted messages, that can only be read by the intended sender and receiver
and neither be read by third parties, nor by Trale itself.

We achieved this by conducting excessive research on encryption algorithms and developing our own handshake for key
exchanges.
Each member of the Trale team managed to enhance their competences in different areas.
Dave Hoevenaars and Maurits van der Zee used their newly achieved project management knowledge to keep the project in
a steady direction and to keep the original vision, whilst also working on the end user frontend.

J\"orn Neumeyer was in charge of the encryption algorithm which formed one of the most important features of the
application.
He in the process became an expert for encryption technology.
Together with Tobias Jansen he worked on the backend services.
Tobias Jansen and Malte Castner were in charge of the admin panel.
Malte would design the frontend and improve his Electron and Angular skills, while Tobias was in charge of the required
backend services and would in the process improve his typescript skills.

Now that the project is over all the aforementioned features are completely implemented and the resulting application
of the software factory can now, thanks to the excessive documentation of the project, function as a solid basis to be
extended upon by future SOFA groups and brings huge potential for both, the upcoming SOFA groups to further their
knowledge and the end user to finally have the secure messaging service everyone promises.
