%!TEX root = ../report.tex
\chapter{Context}\label{ch:context}
In this chapter we will discuss the context of the SocialStuff project sequence.

The Social Stuff project sequence is being carried out in the SoFa module in Fontys Venlo. SoFa is short for Software Factory and the idea of this module is to simulate an authentic software project as you would carry out in your professional career as Software Engineer. The authentic part in that comes from the fact that you have to colaborate in a somewhat bigger project for a real world customer, with the lecturers as your guidance. The goal which this module aims to achieve is that you will be ready for your bachelor thesis and have had a final reharsal before the real deal. As this year SoFa project is early in the Social Stuff project life cycle, we had to do a lot of analysis and research in order to make things work.

When the SoFa module comes to a close, the Social Stuff has quite a few deliverables to produce regarding the project sequence:
1. A group dossier(this document)
2. A personal development plan
3. A pitch video
4. A handover doucment
5. A peer review

All these deliverables are geared towards making a student aware of their role in a software development team, and how to plan for self-improvement.
Subsequently, the pitch video, handover document and peer review teaches you valueble real world skills in terms of reviewing your teammembers, presenting your project in a proffesional and convincing manner.
Finally, you can deliver the project to next years development team in an understable and professional manner using the handover document

Taking all of this into account it makes sense to use a real world project with real world complexity such as Social Stuff for a module that aims to test so many different things.
Do note that the benefits go both ways: not only do the project members learn a lot from developing this project in this setting but it is very good for the project's health as well. This is made possible by lecturer meetings which encourages best practices being upheld when working on a project of this scale

