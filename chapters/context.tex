%!TEX root = ../report.tex
\chapter{Context}\label{ch:context}
The Trale project sequence is being carried out in the \ac{sofa} module at Fontys Venlo.
The idea of this module is to simulate an authentic software project, as you would carry out in your professional career
as a software engineer or project manager.
The authentic part in that comes from the fact that we have to collaborate on a more sophisticated project in terms of
cope and a real world customer.
The goal of this module is to have a final rehearsal of all learnings until today, before our bachelor thesis starts.
As this year's \ac{sofa} project is early in the Trale project life cycle, we had to do a lot of analysis and research
in order to deliver a successful project.

When the \ac{sofa} module comes to its end, the Trale project has quite a few deliverables to produce regarding the
project sequence:

\begin{itemize}\setlength\itemsep{-.5em}
    \item Group report (this document)
    \item Personal development plan
    \item Handover document
    \item Pitch video
\end{itemize}

All these deliverables are geared towards making a student aware of their role in a software development team, and how
to plan for self-improvement.
Subsequently, the pitch video, handover document and peer review teaches you valuable real world skills in terms of
reviewing your team members as well as presenting your project in a professional and convincing manner.
Finally, you can deliver the project to next years' development team in an understandable and professional manner using
the handover document.

Taking all of this into account, it makes sense to use a real world project with real world complexity such as Trale
for a module that aims to test many skills.
This is made possible by lecturer meetings which encourage best practices being upheld when working on a project of
this scale.
