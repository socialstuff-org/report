%!TEX root = ../report.tex


\chapter{Implementation}\label{ch:implementation}

% TODO @all add introduction to implementation
\lipsum[2-4]


\section{Libraries}\label{sec:libraries}
% TODO rename this chapter to make clear that the libraries are self-developed
% TODO @all add introduction to libraries
% -> why libraries?
% -> advantages?
% -> disadvantages?

\subsection{Persistence}\label{subsec:persistence}

% TODO @joernneumeyer & @mauritsvanderzee
% TODO add link to npm

\subsection{Transport}\label{subsec:transport}
% TODO @joernneumeyer & @mauritsvanderzee
% TODO add link to npm

\subsection{Utilities}\label{subsec:utilities}
% TODO @joernneumeyer & @mauritsvanderzee
% TODO add link to npm


\section{Frontend}\label{sec:frontend}

\subsection{Services}\label{subsec:services}

% TODO rework introduction section
The following part of the dossier will describe the different services which have been developed during the time of the project.
This will include a description of each service, followed by an REST API specification.

\subsubsection{API Service}\label{subsubsec:api-service}

The API service is a rather small service located in the desktop client.
The service is responsible for managing the global state of all relevant API information in the desktop application.
In the current state of development this includes the following information

\begin{itemize}\setlength\itemsep{-0.5em}
    \item hostname - the hostname of the server (e.g.\ trale.org)
    \item port - the default port on which the service is listening for incoming connections (default: 8086)
    \item the remote endpoint - a constructed string for building a connection to the server (e.g.\ http://trale.org:8086)
    \item trale port - the default port on which the service is listening for encrypted messages (default: 8087)
\end{itemize}

Further, the service provides several helper methods for retrieving and updating the remote endpoint as well as several
getters and setters for all above mentioned attributes.

\subsubsection{Auth Service}\label{subsubsec:auth-service}
The auth service is responsible for managing authentication related tasks such as login, logout, registration and
minor helper functions (e.g.\ forgot password).

\paragraph{Login function}
The login takes a username and a password as parameters and executes a post request to the remote endpoint specified in
the \textbf{\hyperref[subsubsec:api-service]{API service}}.
On a successful login a token will be provided for future authentication, otherwise an error stating 'Invalid
credentials - Status code 400' will be returned.

% TODO @joernneumeyer maybe add register / logout etc. here as well? refactor code? not in auth service yet

Due to the mentioned responsibilities, this service is only injected in authentication related components.

\subsubsection{Contact Service}\label{subsubsec:contact-service}

The contact service is responsible for managing all contact related tasks and corresponding state management.
It owns a contact array holding all contacts for the currently logged-in user.
On application startup the contact service will load all contacts from drive by utilizing the
\textbf{\hyperref[subsubsec:crypto-storage-service]{Crypto Storage Service}}.

Besides providing a loading wrapper for retrieving encrypted contacts from the drive it also manages actions such as
adding contacts, updating contacts or fetching the latest message of all contacts to display those in the contact list.

Lastly, the contact service provides a function to load the actual chat of a specific contact with the currently
logged-in user.

% TODO @joernneumeyer any additional remarks / information to be added?

\subsubsection{Debug Service?}
% TODO @joernneumeyer & @mauritsvanderzee
% @joernneumeyer shall we include debug service?

\subsubsection{Crypto Storage Service}
% TODO @joernneumeyer & @mauritsvanderzee

\subsubsection{Key Registry Service}
% TODO @joernneumeyer & @mauritsvanderzee

\subsubsection{Titp Service}
% TODO @joernneumeyer & @mauritsvanderzee

\subsubsection{Util Service}
% TODO @mauritsvanderzee

\subsection{Components}\label{subsec:components}


\section{Backend}\label{sec:backend}

\subsection{Database Connections}\label{subsec:database-connections}

For the project we used two database management systems, MongoDB and MySQL, depending on the requirements of the backend
services.

\subsubsection{MongoDB}
MongoDB is a non-relational, document based database management system.
This brings a few advantages ad disadvantages, depending on the use case.
MongoDB is best when you want the flexibility of schema.
You can easily use replica sets with MongoDB and can take advantage of scalability.
Expansion plans are flexible and can be easily achieved by adding more machines and RAM to the system.
It also includes document validations and integrated systems.

A big disadvantage of MongoDB is that over time the data size will increase drastically over time.
Compared to other DBMS systems the speed is comparatively low.

\paragraph{Why and where was MongoDB used?}
Mongo DB was used for the messages.
This is because the messages are non-relational data in the case of SocialStuff.
The disadvantage of low speed once the data amount increases is neglectable in the case of SocialStuff, as the data will
only be stored temporarily on the server.
The data amount is therefore not that high.
Of course local machines will still store the messages on a longer period, however as only the messages with chat
partners are stored the amount of data is still comparatively low.

Due to the high scalability of MongoDB it is excellent for SocialStuff, as the entire system is based on users adding
more servers to the system to increase the security of the system.

\subsubsection{MySQL}
MySQL is a relational database management system, this also features advantages and disadvantages.
For starters MySQL provides a high performance and runs on almost every system which is important for SocialStuff as
users will be able to setup the servers themselves.
Using MySQL will make this process simpler for them.
MySQL also provides a high security, because the database administrator can grant appropriate permissions to
applications that access stored procedures in the database without giving any permissions on the underlying database
tables.
A disadvantage of MySQL is that it becomes slower the more data is stored on it due to a high memory usage.

\paragraph{Why and where was MySQL used?}
MySQL was used for all the relational data, like the data used in the identity service (user data) and reporting
service.
This was done as this data is not mere transactional data (like the messages) but data that is meant to stay.
Furthermore, the high scalability will proof useful once the application will be used by more users and more than our
one test server is online.

\subsection{Services}\label{subsec:services2}

\subsubsection{Authentication Service}
% TODO @joernneumeyer

\subsubsection{File Service}
% TODO @all shall we just briefly mention that this might be a service for future features / development

\subsubsection{Identity Service}
\label{subsubsec:identitySer}
% TODO @joernneumeyer

\subsubsection{Settings Service}\label{subsubsec:settingsSer}

The Settings service provides an interface to the admin panel frontend and carries out the settings to the other
services.
This service was developed in parallel to the admin panel frontend. %TODO @maltecastner insert reference to component, once it has been added
First only the for the UI necessary functionality would be added, later on the actual functionality was added
(e.g.\ that users can actually be reported for the report reasons added in the settings).
As it can be seen from the use case diagram~\ref{fig:ucd} the settings service carries out a few tasks by itself whilst
forwarding certain other requests to the \hyperref[subsubsec:identitySer]{\textbf{identity service}} and the
\hyperref[subsubsec:reportingSer]{\textbf{reporting service}}.
The use cases as well as their descriptions are located in the
\hyperref[ch:software-requirements-specification-(srs)]{\textbf{\ac{srs}}}

\begin{figure}[!ht]
    \centering
    \includegraphics[width=1.0\textwidth]{./images/UseCaseDiagramAdminPanel}
    \caption{Use case diagram of the admin panel showing responsibility of each AP service}
    \label{fig:ucd}
\end{figure}

\paragraph{Rest interface}

The settings service provides the following rest endpoints:

\begin{lstlisting}[label={lst:lstlisting}]
GET: /settings/report-reason
\end{lstlisting}

Returns all report reasons as a JSON array.

\begin{lstlisting}[label={lst:lstlisting2}]
POST: /settings/report-reason
body:
{
"reason": "some reason",
"max_report_violations": 5
}
\end{lstlisting}

Adds a new report reason.

\begin{lstlisting}[label={lst:lstlisting3}]
PUT: /settings/report-reason
body:
{
"id": 123,
"reason": "some reason",
"max_report_violations": 5
}
\end{lstlisting}

Edits an existing report reason.

\begin{lstlisting}[label={lst:lstlisting4}]
DELETE: /settings/report-reason
headers:
- "id": 123
\end{lstlisting}

Deletes a report reason with an id that is provided as a header.

\begin{lstlisting}[label={lst:lstlisting5}]
GET: /settings/security
\end{lstlisting}

Returns the current security settings.

\begin{lstlisting}[label={lst:lstlisting6}]
PUT: /settings/security
body:
{
    "two_factor_auth": {
    "on" : true,
    "phone": false,
    "email": true
},
    "confirmed_emails_only": true,
    "individual_pwd_req": {
    "on": true,
    "upper_case": true,
    "number": true,
    "special_char": true,
    "reg_ex": false,
    "reg_ex_string": "[]"
},
    "inv_only": {
        "on": false,
        "inv_only_by_adm": false
    }
}
\end{lstlisting}

Edits the security settings.
The new settings are provided in body.

\paragraph{Example add report reason}

\begin{figure}[!ht]
    \centering
    \includegraphics[width=1.0\textwidth]{./images/SequenceDiagram_AddReportReason}
    \caption{Use case diagram of the admin panel showing responsibility of each AP service}
    \label{fig:addReportReason}
\end{figure}

The sequence diagram~\ref{fig:addReportReason} shows an example of the communication of the settings service with the
aforementioned services.
The event is triggered by a request from the client.
Firstly the settings service will be verifying the identity of the sender by checking at the identity service, if the
user who sent out the request is in fact an administrator.
Then the request will e forwarded to the reporting service.
Here it will be checked if the request format is valid.
Should this not be the case, the reporting service will return an error (status 400).
Otherwise, the reason will be added in the reporting database, which will then be returned to the settings service,
who will then let the client know by response that the insertion was a success.
This way each service has a clearly defined task.

\subsubsection{Reporting Service}\label{subsubsec:reportingSer}

The reporting service handles the reporting system.
This includes administrative tasks like managing for which reasons users can be reported, as well as the reporting
system itself (reporting users and banning them).
The administrative tasks, like adding new report reasons, manually blocking and unblocking users, can be done via the
\hyperref[subsubsec:settingsSer]{\textbf{settings service}}.
This service was added in the later stages of the project.

First the entire reporting was handled via the \hyperref[subsubsec:settingsSer]{\textbf{settings service}}, however it
was decided that the reporting itself should possess its own service. % TODO @tobiasjansen --> why? explanation missing
Therefore, it was necessary to outsource the report related functionality from the
\hyperref[subsubsec:settingsSer]{\textbf{settings service}} to the
\hyperref[subsubsec:reportingSer]{\textbf{reporting service}}.

\paragraph{Rest Interface}
\begin{lstlisting}[label={lst:lstlisting7}]
POST: /report-reasons
body:
{
    "reason": "some reason",
    "max_report_violations": 5
}
\end{lstlisting}

Adds a new report reason returning the ID of the added report reason.

\begin{lstlisting}[label={lst:lstlisting8}]
PUT: /report-reasons
body:
{
    "id": 12
    "reason": "some reason",
    "max_report_violations": 5
}
\end{lstlisting}

Edits a report reason by the \enquote{ID} key and returns the edited reason.

\begin{lstlisting}[label={lst:lstlisting9}]
GET: /report-reasons
\end{lstlisting}

Returns all existing report reasons as JSON array

\begin{lstlisting}[label={lst:lstlisting10}]
DELETE: /report-reasons
headers
    - id: 12
\end{lstlisting}

Deletes a report reason by ID which is provided in the header of the request.
Returns the ID of the deleted request in its body.

\begin{lstlisting}[label={lst:lstlisting11}]
Headers:
    - user_token
Body:
{
    "username": "userHashOfUserBeingReported",
    "reason_id": 123
}
\end{lstlisting}

Reports a user based of the report reason, and the users' username hash.
A user can only be reported by the same user for the same reason after 15 minutes to prevent spamming.
Otherwise, the report will be processed by the backend, and the user will be set to \enquote{blocked} in case one
exceeds the maximum report violation counter.

