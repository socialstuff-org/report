%!TEX root = ../report.tex
\chapter{Implementation}\label{ch:implementation}

% TODO @all add introduction to implementation
\lipsum[2-4]

\section{Libraries}\label{sec:libraries}
% TODO rename this chapter to make clear that the libraries are self-developed
% TODO @all add introduction to libraries
% -> why libraries?
% -> advantages?
% -> disadvantages?

\section{Frontend}\label{sec:frontend}

\subsection{Services}\label{subsec:services}

\subsubsection{API Service}\label{subsubsec:api-service}
The API service is a rather small service located in the desktop client.
The service is responsible for managing the global state of all relevant API information in the desktop application.
In the current state of development this includes the following information

\begin{itemize}
    \item hostname - the hostname of the server (e.g. trale.org)
    \item port - the default port on which the service is listening for incoming connections (default: 8086)
    \item the remote endpoint - a constructed string for building a connection to the server (e.g.\ http://trale.org:8086)
    \item trale port - the default port on which the service is listening for encrypted messages (default: 8087)
\end{itemize}

Further, the service provides several helper methods for retrieving and updating the remote endpoint as well as several
getters and setters for all above mentioned attributes.

\subsubsection{Auth Service}\label{subsubsec:auth-service}
The auth service is responsible for managing authentication related tasks such as login, logout, registration and
minor helper functions (e.g.\ forgot password).

\paragraph{Login function}
The login takes a username and a password as parameters and executes a post request to the remote endpoint specified in
the \textbf{\hyperref[subsubsec:api-service]{API service}}.
On a successful login a token will be provided for future authentication, otherwise an error stating 'Invalid
credentials - Status code 400' will be returned.

% TODO @joernneumeyer maybe add register / logout etc. here as well? refactor code? not in auth service yet

Due to the mentioned responsibilities, this service is only injected in authentication related components.

\subsubsection{Contact Service}\label{subsubsec:contact-service}

The contact service is responsible for managing all contact related tasks and corresponding state management.
It owns a contact array holding all contacts for the currently logged-in user.
On application startup the contact service will load all contacts from drive by utilizing the
\textbf{\hyperref[subsubsec:crypto-storage-service]{Crypto Storage Service}}.

Besides providing a loading wrapper for retrieving encrypted contacts from the drive it also manages actions such as
adding contacts, updating contacts or fetching the latest message of all contacts to display those in the contact list.

Lastly, the contact service provides a function to load the actual chat of a specific contact with the currently
logged-in user.

% TODO @joernneumeyer any additional remarks / information to be added?

\subsubsection{Debug Service?}
% TODO @joernneumeyer & @mauritsvanderzee
% @joernneumeyer shall we include debug service?

\subsubsection{Crypto Storage Service}
% TODO @joernneumeyer & @mauritsvanderzee

\subsubsection{Key Registry Service}
% TODO @joernneumeyer & @mauritsvanderzee

\subsubsection{Titp Service}
% TODO @joernneumeyer & @mauritsvanderzee

\subsubsection{Util Service}
% TODO @mauritsvanderzee

\subsection{Components}

\section{Backend}

\subsection{Database Connections}

\subsection{Services}

\subsubsection{Authentication Service}
% TODO @joernneumeyer

\subsubsection{File Service}
% TODO @all shall we just briefly mention that this might be a service for future features / development

\subsubsection{Identity Service}
% TODO @joernneumeyer

\subsubsection{Settings Service}
% TODO @maltecastner & tobiasjansen
