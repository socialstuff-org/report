%!TEX root = ../report.tex

\chapter{Project Management}\label{ch:project-management}

\section{Project Charter}\label{sec:project-charter}

\begin{table}[]
    \begin{tabular}{ll}
        Project Name & Trale / SocialStuff (former working title) \\
        Submitted by \& date & Dave Hoevenaars, Maurits van der Zee, 28-09-2020 \\
        Strategic Objective & Providing privacy in a centralized social media market. \\
        Project ID & SOFA2020-HOM02-SocialStuff \\
        Confidentiality & N/A
    \end{tabular}
    \caption{Project Charter Overview}
    \label{tab:my-table}
\end{table}

\subsection{Problem Statement}\label{subsec:problem-statement}

Centralized Social Media platforms has been dominating the market for the last decade, this is a problem because nearly
everybody who is communicating throughout the internet is relaying on these businesses, and they’re business model
revolves around selling your private data.
This can be especially damaging for high-profile individuals which reputation or credibility could be damages by
providing such data.
Additionally, the large corporations that host such services are based in countries which can force said businesses to
provide access to their systems (USA, China \& Russia).

\subsection{Objective}\label{subsec:objective}

The goal of the Social Stuff project sequence is to develop a working prototype of the decentralized chat application
(front end \& backend) as well as supporting documentation.
This will be done by 20th January 2021.

\subsection{Critical Success Factors}\label{subsec:critical-success-factors}

In order to prove Social Stuff’s claim of privacy and security, a penetration test should be carried out, and the
results should be divided in 5 sub-sections: Reconnaissance, Scanning, Gaining Access, Maintaining Access and Covering
tracks.
In these categories a risk analysis will be carried out and if the exposure is not greater than 2, the penetration test
will consider a pass.

\subsection{Risks}\label{subsec:risks}

In the Social Stuff project we identified a few risks which may occur such as:

\begin{itemize}
    \item Not achieving customer satisfaction due to not meeting the client’s requirements
    \item Employee capacity decrease because of illness
    \item Stretching the project’s time scope due to unfamiliarity with programming languages and/or technologies
\end{itemize}

\subsection{Scope focus}

In scope of the Social Stuff project sequence is: Creation of prototype of decentralized structure of Social Stuff

\begin{itemize}
    \item A report explaining analysis, design and implementation choices
    \item Creation of front-end of Social Stuff
    \item Analysis, Design and in-code documentation artefacts
    \item Optional: Customer Website
\end{itemize}

However, in this project sequence we will not have a fully released version of SocialStuff and the chat application
will not be deployed either, the main goal is to have a working prototype, in order to showcase the potential of the
project sequence.

\subsection{Key activities \& dates}\label{subsec:key-activities-and-dates}

\begin{table}[]
    \begin{tabular}{ll}
        Activity                & Date       \\
        Project Kick-off        & 02-9-2020  \\
        Business case           & 01-10-2020 \\
        Project Management Plan & 01-10-2020 \\
        Project Charter         & 15-10-2020 \\
        Project Scope Baseline  & 25-10-2020 \\
        WBS                     & 30-10-2020 \\
        Project Poster          & 06-01-2021 \\
        Project Handover        & 20-01-2021
    \end{tabular}\label{tab:table2}
\end{table}

\subsection{Deliverables}\label{subsec:deliverables}

\begin{itemize}
    \item Creation of prototype of decentralized structure of Social Stuff
    \item A report explaining analysis, design and implementation choices
    \item Creation of front-end of Social Stuff
    \item Analysis, Design and in-code documentation artefacts
    \item Optional: Customer Website
\end{itemize}

\subsection{Business Case}

The business case in non-financial;
providing customer satisfaction by delivering a social media platform which is excellent in its privacy standards.

\subsection{Proposed start \& end dates}\label{subsec:proposed-start-and-end-dates}

The Social Stuff project sequence will start at \textbf{2-9-2020} and will end \textbf{20-1-2020}.

\subsection{Stakeholders \& Resources}\label{subsec:stakeholders-and-resources}

The resources we will utilize for the completion of the SocialStuff project sequence includes:
\begin{enumerate}
    \item Trello, for a Kanban style agile project management
    \item A shared repository on GitHub to do work simultaneously
    \item MS Teams platform for general communication and team meetings
    \item Adobe XD for UI design
    \item Coding Standards
    \item The Source Code
\end{enumerate}

Furthermore, to support the completion of the project sequence the following stakeholders are involved:
\begin{itemize}
    \item Gerhard Bongardt: Business Owner
    \item Pieter van den Hombergh: Project Governance
    \item Jörn Neumeyer: Product Owner
\end{itemize}

