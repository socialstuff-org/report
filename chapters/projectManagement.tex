%!TEX root = ../report.tex


\chapter{Project Management}\label{ch:project-management}


\section{Project Charter}\label{sec:project-charter}

\begin{table}[htp]
    \centering
    \caption{Project Charter General Details}
    \begin{tabular}{@{}|l|l|@{}}
        \hline
        \textbf{Activity} & \textbf{Date} \\
        \toprule
        \hline
        Project Name & Trale / SocialStuff (former working title) \\
        Submitted by \& date & Dave Hoevenaars, Maurits van der Zee, 28-09-2020 \\
        Strategic Objective & Providing privacy in a centralized social media market. \\
        Project ID & SOFA2020-HOM02-SocialStuff \\
        Confidentiality & N/A \\
        \hline
    \end{tabular}
    \label{tab:my-table}
\end{table}

\subsection{Problem Statement}\label{subsec:problem-statement}

Centralized Social Media platforms has been dominating the market for the last decade, this is a problem because nearly
everybody who is communicating throughout the internet is relaying on these businesses, and they’re business model
revolves around selling your private data.
This can be especially damaging for high-profile individuals which reputation or credibility could be damages by
providing such data.
Additionally, the large corporations that host such services are based in countries which can force said businesses to
provide access to their systems (USA, China \& Russia).

\subsection{Objective}\label{subsec:objective}

The goal of the Social Stuff project sequence is to develop a working prototype of the decentralized chat application
(front end \& backend) as well as supporting documentation.
This will be done by 20th January 2021.

\subsection{Critical Success Factors}\label{subsec:critical-success-factors}

In order to prove Social Stuff’s claim of privacy and security, a penetration test should be carried out, and the
results should be divided in 5 sub-sections: Reconnaissance, Scanning, Gaining Access, Maintaining Access and Covering
tracks.
In these categories a risk analysis will be carried out and if the exposure is not greater than 2, the penetration test
will consider a pass.

\subsection{Risks}\label{subsec:risks}

In the Social Stuff project we identified a few risks which may occur such as:

\begin{itemize}
    \item Not achieving customer satisfaction due to not meeting the client’s requirements
    \item Employee capacity decrease because of illness
    \item Stretching the project’s time scope due to unfamiliarity with programming languages and/or technologies
\end{itemize}

\subsection{Scope focus}

In scope of the Social Stuff project sequence is: Creation of prototype of decentralized structure of Social Stuff

\begin{itemize}
    \item A report explaining analysis, design and implementation choices
    \item Creation of front-end of Social Stuff
    \item Analysis, Design and in-code documentation artefacts
    \item Optional: Customer Website
\end{itemize}

However, in this project sequence we will not have a fully released version of SocialStuff and the chat application
will not be deployed either, the main goal is to have a working prototype, in order to showcase the potential of the
project sequence.

\subsection{Key activities \& dates}\label{subsec:key-activities-and-dates}

\begin{table}[htp]
    \centering
    \begin{tabular}{@{}|l|l|@{}}
        \hline
        \textbf{Activity}       & \textbf{Date} \\
        \toprule
        \hline
        Project Kick-off        & 02-9-2020     \\
        Business case           & 01-10-2020    \\
        Project Management Plan & 01-10-2020    \\
        Project Charter         & 15-10-2020    \\
        Project Scope Baseline  & 25-10-2020    \\
        WBS                     & 30-10-2020    \\
        Project Poster          & 06-01-2021    \\
        Project Handover        & 20-01-2021    \\
        \hline
    \end{tabular}\label{tab:table2}
\end{table}

\subsection{Deliverables}\label{subsec:deliverables}

\begin{itemize}
    \item Creation of prototype of decentralized structure of Social Stuff
    \item A report explaining analysis, design and implementation choices
    \item Creation of front-end of Social Stuff
    \item Analysis, Design and in-code documentation artefacts
    \item Optional: Customer Website
\end{itemize}

\subsection{Business Case}

The business case in non-financial;
providing customer satisfaction by delivering a social media platform which is excellent in its privacy standards.

\subsection{Proposed start \& end dates}\label{subsec:proposed-start-and-end-dates}

The Social Stuff project sequence will start at \textbf{2-9-2020} and will end \textbf{20-1-2020}.

\subsection{Stakeholders \& Resources}\label{subsec:stakeholders-and-resources}

The resources we will utilize for the completion of the SocialStuff project sequence includes:
\begin{enumerate}
    \item Trello, for a Kanban style agile project management
    \item A shared repository on GitHub to do work simultaneously
    \item MS Teams platform for general communication and team meetings
    \item Adobe XD for UI design
    \item Coding Standards
    \item The Source Code
\end{enumerate}

Furthermore, to support the completion of the project sequence the following stakeholders are involved:
\begin{itemize}
    \item Gerhard Bongardt: Business Owner
    \item Pieter van den Hombergh: Project Governance
    \item Jörn Neumeyer: Product Owner
\end{itemize}


\section{Project Management Plan}\label{sec:project-management-plan}

\subsection{Management Summary}\label{subsec:management-summary}

The joint collaboration between: Gedak GmbH, SocialStuff and the SoFa Fontys students is resulting in to a project
sequence.
The final goal is providing an implementation of an encrypted chat platform which provides privacy for its users by
providing a decentralized structure in which anonymity can be guaranteed.
This project is first in the complete project sequence.
This means that we have to do a lot of analysis to understand and determine how the solution will look like.
This especially goes for the encryption techniques.
The current project sequence is mainly focused on providing a working prototype and providing thorough documentation
of said prototype.
As this project’s affinity is very close to some Fontys students working on this project, we will have an
excellent starting point for us to provide documentation which will prove invaluable for future iterations of the
project.
As this project is part of sequence, it is important to determine what the scope and justification of this iteration
will be.
The project's scope and justification will be discussed in this document.

\subsection{Introduction}\label{subsec:introduction}
This document will provide an overview over the Social Stuff project.
To achieve this, the project will briefly explain the project itself, what the final products will be and who the
stakeholders are.
Afterwards the document will address, which potential risks could occur during the development and how the team will
deal with them.
Next it will be addressed, how the change management will be handled during production and how a high quality can be
ensured.
Next the project methodology will be handled.
Finally, the communication plan will be introduced.

\subsection{SocialStuff}\label{subsec:socialstuff}

This chapter explains the company briefly;
it also describes the problem description and our tasks within the duration of the project.

\subsubsection{Company description}

GEDAK is an IT company which supports its customers with practical solutions for marketing, sales and customer service.
Their work focuses on web-based projects such as store systems, sales and CRM solutions, reporting applications as well
as ERP solutions.
The office is located near Düsseldorf (DE) with about 50 employees who are supporting projects at large and medium-sized
companies.

\subsubsection{Problem description}

Social Stuff is a free-software project and is intended to offer an open, decentralized alternative to current
proprietary and centralized communication and social-media platforms.
The decentralized model allows for an independent utilization of the software, providing companies (or users in general)
with more sovereignty and less dependence on large service providers.

\subsubsection{Our Task}

Our task is to analyse current communication services and how a decentralized approach would disrupt the market by
providing the same features with increased focus on privacy and security.
Afterwards, we will design how the new decentralized system could look like.
Lastly, the system shall be implemented using an agile approach.
During the project the whole project should be managed and monitored properly as well as documented in a detailed and
structured manner.

\subsection{Project justification}\label{subsec:project-justification}

\subsubsection{Project Objective}

The goal of the Social Stuff project, is to Analyse, Design, Implement \& Test an encrypted chat application which
resources will be distributed in a decentralized manner, by the users as they have to set up server themselves.
This application will be deployed in a highly competitive market, which means that Social Stuff will have to
differentiate itself in order to be successful.
This is achieved by the decentralized nature of the application which provides users near full control of the
chat application and various privacy settings.
However, this alone might not be enough, to set Social Stuff aside from the competition.
So, an optional goal be to find an additional way of provide differentiation between Social Stuff and its competitors,
without forgoing the previously mentioned perquisites.

\subsubsection{Project requirements \& resources}

For the project we will need several resources and there are some high-level requirements for this project.

\paragraph{Resources}
\begin{itemize}
    \item Trello, for a Kanban style agile project management
    \item A shared repository on GitHub to do work simultaneously
    \item MS Teams platform for general communication and team meetings
    \item Adobe XD for UI design
    \item Coding Standards
    \item The Source Code
\end{itemize}

\paragraph{Required knowledge areas}
\begin{itemize}
    \item Programming Languages (TypeScript)
    \item Frameworks (Express, Angular, UI frameworks (e.g.\ Angular Material))
    \item Database technologies (SQL (MariaDB), noSQL (MongoDB))
    \item Agile development (Kanban)
    \item Encryption (RSA, AES, Diffie Hellman (Elliptic curves))
    \item Version control (Git)
\end{itemize}

\subsubsection{Project Scope}

The project scope will describe the activities we will be carrying out but also the activities that will be beyond the
scope of this project.

\paragraph{In scope}
\begin{itemize}
    \item Creation of prototype of decentralized structure of the Social Stuff
    \item A report explaining our analysis, design and implementation choices
    \item Creation of front-end of Social Stuff
    \item Analysis, Design and in-code documentation
    \item Optional: Customer Website
\end{itemize}

\paragraph{Out of scope}
\begin{itemize}
    \item A fully released version of SocialStuff
    \item Deploying Social Stuff
\end{itemize}

\subsection{Project Products}

\subsubsection{Customer quality expectations}

The customer expects high-quality software solutions for decentralized communication via one or multi server network.
The client expects at least a desktop client.
However, as we are using cross-platform technologies we try to reuse most of the desktop client solution to implement
a mobile client as well.

\subsubsection{Customer acceptance criteria}

In this section, we will mention the deliverables and which purpose they serve for.

\begin{table}[htp]
    \caption{List of deliverables and their corresponding acceptance criteria as well as serving purpose}
    \begin{tabular}{@{}|l|l|l|@{}}
        \toprule
        \hline
        \textbf{Deliverable} &
        \textbf{Acceptance Criteria} &
        \textbf{Product Goal} \\
        \hline
        \hline
        Documentation &
        \begin{tabular}[c]{@{}l@{}}Detailed, \\ structured, \\ consistent, \\ cohesive, easy \\ understandable\end{tabular} &
        \begin{tabular}[c]{@{}l@{}}Enables easy extensibility by providing \\ future developer teams to improve \\ certain parts and services of the solution\end{tabular} \\
        \begin{tabular}[c]{@{}l@{}}Models (part of \\ documentation)\end{tabular} &
        \begin{tabular}[c]{@{}l@{}}All models are \\ written in valid \\ UML\end{tabular} &
        \begin{tabular}[c]{@{}l@{}}Make the underlying technologies easier \\ to understand\end{tabular} \\
        \begin{tabular}[c]{@{}l@{}}OOS: Server \\ installation\\ package\end{tabular} &
        \begin{tabular}[c]{@{}l@{}}Package should be \\ easy deployable on \\ a server of the \\ user's choice\end{tabular} &
        \begin{tabular}[c]{@{}l@{}}Server package should enable individuals \\ to host their own communication server \\ by making use of docker\end{tabular} \\
        Desktop Client &
        \begin{tabular}[c]{@{}l@{}}Fully functional \\ desktop client \\ which should be \\ able to register \\ and login to a \\ server and \\ communicate with \\ several chat \\ partners either on \\ the same server or \\ a different server\end{tabular} &
        \begin{tabular}[c]{@{}l@{}}Deliver a desktop client to individuals \\ which can easily be used to communicate \\ privately and securely\end{tabular} \\
        OOS: Mobile Client &
        \begin{tabular}[c]{@{}l@{}}Fully functional \\ mobile client \\ which should be \\ able to register \\ and login to a \\ server and \\ communicate with \\ several chat \\ partners either on \\ the same server or \\ a different server\end{tabular} &
        \begin{tabular}[c]{@{}l@{}}Deliver a mobile client to individuals \\ which can easily be used to communicate \\ privately and securely\end{tabular} \\
        Optional: Website &
        \begin{tabular}[c]{@{}l@{}}Deployable \\ website for \\ trale.org\end{tabular} &
        \begin{tabular}[c]{@{}l@{}}Providing a user-friendly website for two \\ main purposes:		\\ - Provide potential users information \\ about the service and how SocialStuff \\ works (should be easily understandable, \\ not to tech sided) 			\\ - Provide (potential) users a download \\ portal for the server as well as the \\ desktop and mobile clients\end{tabular} \\
        Project Dossier &
        \begin{tabular}[c]{@{}l@{}}Detailed, \\ structured, \\ consistent, \\ cohesive, easy \\ understandable\end{tabular} &
        \begin{tabular}[c]{@{}l@{}}Document SOFA module for \\ reporting purposes\end{tabular} \\
        &
        &
        \\ \hline
    \end{tabular}\label{tab:table3}
\end{table}

\subsubsection{Product Opportunities}

Aside from the project goals, Social Stuff has to motivation to become something more than just a one-off project.
As this project sequence will only last us to January, it’s important to start thinking about future iterations of the
project sequence and what feature those sequences might bring forth.
In order to ensure reusability of project sequence artefacts: we will maintain that all documents will be in English.
The main goal is to develop a prototype chat application, which can address the consumer demand for high privacy
communication technologies.
One could argue however that in such a competitive market, it is difficult to set your product aside from the
competition.
Therefore, further project sequences could include: expanding the platform in terms of features, maximizing security,
maximizing adaptability, or any other way of differentiating SocialStuff.

\subsection{Project controls}

This section of the document will clarify how the project will be managed.
Next the risk management will be addressed.

\subsubsection{Stakeholder management}

Stakeholder management is a critical component to the successful delivery of any project.
A stakeholder is any individual, group or organization that can affect the project.
In the table below, you can find all the stakeholders of our project.
The name, role, how much influence do they have from low to high (1, 2, 3), the stakeholders most important goal,
contribution and the best way to manage.

The following table displays each stakeholder and their corresponding roles, influences and goals and contribution to
the project.

% Please add the following required packages to your document preamble:
% \usepackage{booktabs}
\begin{table}[htp]
    \caption{List of involved stakeholders, their roles and influence on project}
    \begin{tabular}{@{}|l|l|l|l|l|@{}}
        \toprule
        Stakeholder &
        Role &
        Influence &
        \begin{tabular}[c]{@{}l@{}}Most important\\ goal\end{tabular} &
        \begin{tabular}[c]{@{}l@{}}How will\\ he/she\\ contribute\end{tabular} \\ \midrule
        \begin{tabular}[c]{@{}l@{}}Jörn\\ Neumeyer\end{tabular} &
        \begin{tabular}[c]{@{}l@{}}Product Owner\\ Backend Dev\end{tabular} &
        3 &
        \begin{tabular}[c]{@{}l@{}}Properly\\ communicate\\ the\\ requirements \&\\ developing\\ back-end\end{tabular} &
        \begin{tabular}[c]{@{}l@{}}High, \\ 3 times a \\ week active\end{tabular} \\
        \begin{tabular}[c]{@{}l@{}}Dave\\ Hoevenaars\end{tabular} &
        \begin{tabular}[c]{@{}l@{}}Documentation\\ Quality Assurance\\ Front-end dev\end{tabular} &
        2 &
        \begin{tabular}[c]{@{}l@{}}Make sure the\\ project runs \\ smooth/time\\ management \&\\ developing\\ front-end\end{tabular} &
        \begin{tabular}[c]{@{}l@{}}High,\\ 3 times a\\ week active\end{tabular} \\
        \begin{tabular}[c]{@{}l@{}}Tobias\\ Jansen\end{tabular} &
        Back-end dev &
        2 &
        \begin{tabular}[c]{@{}l@{}}Developing\\ back-end\end{tabular} &
        \begin{tabular}[c]{@{}l@{}}High, \\ 3 times a \\ week active\end{tabular} \\
        \begin{tabular}[c]{@{}l@{}}Malte\\ Castner\end{tabular} &
        Front-end dev &
        2 &
        \begin{tabular}[c]{@{}l@{}}Developing\\ front-end\end{tabular} &
        \begin{tabular}[c]{@{}l@{}}High, \\ 3 times a \\ week active\end{tabular} \\
        \begin{tabular}[c]{@{}l@{}}Maurits\\ van der Zee\end{tabular} &
        \begin{tabular}[c]{@{}l@{}}Project \\ management \&\\ back-end dev \& \\ front-end dev\end{tabular} &
        2 &
        \begin{tabular}[c]{@{}l@{}}Make sure the \\ project runs \\ smooth / time\\ management\\ developing\\ front-end and\\ back-end\end{tabular} &
        \begin{tabular}[c]{@{}l@{}}High,\\ 3 times a\\ week active\end{tabular} \\
        Gedak GmbH &
        Client &
        1 &
        \begin{tabular}[c]{@{}l@{}}Properly \\ communicate the\\ requirements / \\ make sure we\\ stay on track\end{tabular} &
        \begin{tabular}[c]{@{}l@{}}Needs to be\\ informed\\ about project\\ status\end{tabular} \\
        \begin{tabular}[c]{@{}l@{}}Fontys Tutor\\ Pieter\\ van den Hombergh\end{tabular} &
        Supervisor &
        1 &
        Supervising the project &
        \begin{tabular}[c]{@{}l@{}}Little but\\ wants to be\\ kept in loop\\ for project\\ status updates\end{tabular}
    \end{tabular}\label{tab:table6}
\end{table}

\subsubsection{Risk management}

The following section will address the potential risks which might occur during the development.
For risk management there will be a list containing all the risks we have identified.
Together with these risks we will also come up with counter measures to either lower the likelihood of occurrence or
lower the impact if said risk occurs.
The risks will be ranked by its exposure, calculated by using the following formula:

\begin{equation}
    Likelihood*Impact=Risk Exposure\label{eq:risk_exposure}
\end{equation}

Each risk will have a ranking concerning its likelihood and impact (1 – 5).
The higher the exposure, the more attention must be paid to said risk.

% Please add the following required packages to your document preamble:
% \usepackage{booktabs}
\begin{table}[htp]
    \caption{List of identified risks related to the project}
    \begin{tabular}{@{}|l|l|l|l|@{}}
        \toprule
        \textbf{ID} &
        \textbf{Risk} &
        \textbf{Consequence} &
        \textbf{Counter measure} \\
        \hline
        R1 &
        \begin{tabular}[c]{@{}l@{}}Due to the lack of \\
                                    knowledge of coding language,\\
                                    spending too much time on \\
                                    researching\end{tabular} &
        \begin{tabular}[c]{@{}l@{}}Might not be able to finish \\
                                    the prototype on time\end{tabular} &
        \begin{tabular}[c]{@{}l@{}}Make sure the proper resources \\
                                    are available to use when it is \\
                                    clear which language, we will be \\
                                    using\end{tabular} \\
        R2 &
        \begin{tabular}[c]{@{}l@{}}End product not satisfying the \\ requirements of the client\end{tabular} &
        \begin{tabular}[c]{@{}l@{}}Client not satisfied with the \\result which will lead to a \\failed project\end{tabular} &
        \begin{tabular}[c]{@{}l@{}}Make sure to show the progress \\and ask for feedback by holding \\demos to the client regularly\end{tabular} \\
        \hline
    \end{tabular}\label{tab:table5}
\end{table}

\begin{table}[htp]
    \caption{Mapping of risks and their corresponding exposures}
    \center
    \begin{tabular}{@{}|l|l|l|l|@{}}
        \toprule
        \textbf{Risk ID} & \textbf{Likelihood} & \textbf{Impact} & \textbf{Exposure} \\
        \hline
        R1      & 2          & 4      & 8        \\
        R2      & 2          & 5      & 10 \\
        \hline
    \end{tabular}\label{tab:table4}
\end{table}

\subsection{Communication Plan}

\subsubsection{Internal}

The internal documentation will be in English.
The main internal communication platform that will be used during this project is Microsoft (MS) Teams.
In MS teams it is possible to chat with each other and share documents.
Our Kanban task planning, which is updated online in Trello, is integrated in MS Teams.
Next to MS teams, we have a WhatsApp chat group for small talks.
For urgent matters we can call each other.

% TODO add table/list of contacts and email addresses here

Next to these communication platforms, we’ll have a daily stand-up with the project team.
Our SOFA-coach is Pieter van den Hombergh.
He will join the weekly project meeting on Tuesday.
Besides, he will be available to answer our questions.

\subsubsection{External}

The external communication will mainly be by E-mail.
Our main contact person from Gedak Gmbh is Gerhard Bongardt.
There will be meetings with Mr. Bongardt after each sprint and additional meetings when necessary.
If there are any questions, Mr. Bongardt will forward the questions to the project manager.
