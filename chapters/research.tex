%!TEX root = ../report.tex
\chapter{Research}\label{ch:research}

As we want to create a secure, decentralized, and privacy focused chat application we needed to execute some research on
security topics.
In the context of a chat application which aims to fulfill those requirements we mainly needed to focus on two key
principles:

\begin{itemize}
    \setlength\itemsep{-.5em}
    \item End-to-end encryption of messages
    \item Verification of a sender's identity
\end{itemize}

\section{Problem Statements}\label{sec:problem-statements}

To be compliant with those two principles we needed research answers to the following problem statements:
\begin{itemize}
    \setlength\itemsep{-.5em}
    \item How to set up end-to-end encryption between two parties who have never met?
    \item How to verity that a message is really from a certain sender?
\end{itemize}

\section{Research Strategy}\label{sec:research-strategy}
Our research strategy was to research which encryption algorithms are present and used in the modern information
technology sector.
Which algorithms serve a high-security standard as well as great usability, and which algorithms are fitting for which
purposes.

\section{Findings}\label{sec:findings}
First, we tried to find out how end-to-end encryption works nowadays.
Our goal was to discover which components are required so that only the recipient of a message is able to decrypt \texttt{it}

\paragraph{RSA}
\ac{rsa} is an asymmetric crypto system which allows for encryption and decryption by means of public and private keys.
With this crypto system it is possible to create signatures of data, which allow for the verification of the origin of
a piece of data, and to verify that it has not been tampered with.
With this setup we can ensure a secure end-to-end encryption procedure.
However, a secure key calculation is relatively time expensive but would be necessary to ensure security if the key
is used for frequent encryption and decryption~\cite{rsa}.
In addition to the time-consuming key generation encryption and decryption are also slower than what symmetric
cipher have to offer.
This is especially disadvantageous when it comes to encryption in the cloud, as server utilization directly correlates
with expenses.
In addition, the amount of data which can be encrypted using \ac{rsa} is directly limited by the key size.
E.g.\ with a 4096bit key we can encrypt data of approximately 512 bytes.
As this filesize is not sufficient for a chat application, this crypto system alone would not be the final solution.

\paragraph{AES}
In contrast to \ac{rsa}, \ac{aes} is a symmetric cipher and not an entire crypto system.
It is a modern standard for data encryption which is also supported on a wide range of hardware.
Key sizes can be chosen between 128, 192, and 256 bits which are all unbreakable with current technology.
However, if upcoming technology might lower the keyspace of the encryption significantly (i.e.\ 128bit to 64bit),
the encryption could be bruteforced.
Therefore, we went for 256bit keys for our encryption.
In contrast to \ac{rsa} with \ac{aes} we do not encounter a limit on the size of the data we want to encrypt.
But the requirement of a symmetric cipher is whoever decrypt a given piece of data, also has to be in possession of the
same key which has been used to encrypt the data~\cite{aes}.
This introduces our next question: How can we establish a common key between to parties via an unsecure connection?

\paragraph{ECDH}
\ac{ecdh} is an asymmetric key exchange algorithm by which two parties may negotiate a common secret which may be used
for symmetric encryption (e.g. \ac{aes}).
It offers improved security over regular Diffie-Hellman (DH), and it does not rely on regular clock arithmetic, but a
variation based on elliptic curves (EC).
The computationally more expensive elliptic curve arithmetic allows for shorted keys which provide the same security as
larger keys in regular Diffie-Hellman.
The calculation of the said shared secret works by mixing key $Priv_A$ with key $Pub_B$ which will result in the same
key as by mixing $Priv_B$ with key $Pub_A$ (\ref{fig:figure46}).
However, only mixing $Pub_A$ with $Pub_B$ will result in an invalid key so by monitoring the public keys it is not
possible to regenerate the shared secret (\ref{fig:figure46}).

\begin{figure}[H]
    \centering
    \includegraphics[width=1.0\textwidth]{./graphics/encryptionPattern}
    \caption{ECDH key exchange illustrated by mixing colors}
    \label{fig:figure46}
\end{figure}

One problem that is not been solved by \ac{ecdh} is the verification, that a certain public key actually belongs to the
one who claims to send it.
But since \ac{rsa} offers such a mechanism, it is used to compromise for that weakness.

\section{Final Cryptographic Setup}\label{sec:final-cryptographic-setup}

After taking the knowledge described into the upper section into account, we came up with the following cryptographic
setup:

\begin{figure}[H]
    \centering
    \includegraphics[height=.8\textheight]{./graphics/encryption}
    \caption{Example key exchange between Alice and Bob}
    \label{fig:figure45}
\end{figure}

\ac{ecdh} public and private keys are used to establish a shared secret.
In order to verify the validity of that secret, \ac{rsa} signatures are used to verify the identity of both Alice and
Bob.
After establishing a common shared secret, Alice and Bob will use \ac{aes} to send data in an encrypted and performant
way to one another.
This way we established an end-to-end encrypted communication channel which is secured against any monitoring of the
conversation content.
