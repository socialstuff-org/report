%!TEX root = ../report.tex
\chapter{Technology Stack}\label{ch:technology-stack}
Finding the right technology stack for a project is a major part of the overall process of realizing a project.
Multiple considerations should be taken into account which we will cover in this section.
This section is a detailed view on the technology stack we have utilized in the project and why we used it.

\section{Keep it simple and agile}\label{sec:keep-it-simple-and-agile}

Finding a proper technology stack starts with deciding a fitting solution for the actual context in which the project
should be realized.
As the project will be developed in a short-term project of 5 months within software factory (SOFA) at Fontys Venlo we
had to go with an agile as well as easy to start with technology.
This enables all team members to get started quickly as well as to adapt accordingly if something does not work out as
planned.

We started with a small concept, and some sketches of how the product could look like.
The actual set of features was not fixed at this point in time.
We knew that the project scope would change during the project phase.
That's why it is even more important to choose for an agile technology.

\section{Project requirements}\label{sec:project-requirements}

As the project is focussing on secure as well as fast communication between several users across the internet, we wanted
to choose a technology stack that supports various system platforms as well as easy migration possibilities for a later
stage of the project.
This would allow us to adapt our developed parts to be translated to further platforms and devices.
As more than 60\% of all users are accessing the internet via mobile devices the focus should also lie on technologies
which are fully supported by mobile platforms and devices.

As security and privacy are our core principles we needed to take this into consideration for our technology stack.
Our focus was set to various technologies which have been around for quite some time already, so we can be sure that
major security vulnerabilities have been eliminated beforehand.

\section{Open-source technologies}\label{sec:open-source-technologies}

Going with open-source technologies provides multiple benefits.
First it enables to project team to focus on the business part of the application.
Most of the time, there is already a good open-source solution for general tasks like front-end frameworks, database
systems or encryption algorithms.
Those implementations enable the team to safe a lot of time as the components do not have to be developed by ourselves.
Last but not least are those already implemented solutions more secure as more people have their eyes on it and report
and fix bugs.

\section{Ecosystem}\label{sec:ecosystem}

As a result of the larger community of open-source technologies a better ecosystem will go along with it.
Important for using open-source software is, that it is properly documented and that is already known in the community.
This helps with later bug fixings and solution findings.
The more people using a technology the more questions and answers will be available on sites like Stackoverflow or
Github forums.

Furthermore, there will be more third-party components and solutions available
(for example via npm (node package manager)).

\section{Long-term trends and support}\label{sec:long-term-trends-and-support}

Lastly, a detailed look on the long-term development of available technologies should be considered. As said before we
should aim for mature technologies as they are more robust, reliable and most of the time also more secure which is
crucial for this project specifically.

\section{Final technology stack}\label{sec:final-technology-stack}

Taking all the above listed considerations into account we have come up with the following technology stack for our
SOFA project:

\subsection{Frontend}\label{subsec:frontend}

\begin{itemize}
    \item \textbf{Electron:} We decided to go with Electron as it enables us to develop one solution which can be
        deployed across several platforms including Windows, macOS and Linux. As Electron is not that difficult to
        include into a default setup we decided to utilize the technology as this increases the target group
        significantly.
    \item \textbf{Angular:} We decided to go with Angular as major front-end framework. It enables us to handle data
        efficiently between several components. Furthermore, it comes with useful features, for example to dynamically
        re-render views if the displayed data changes.
    \item \textbf{Angular Material:} Angular Material provides us with a comprehensive UI library. The library ships
        with preconfigured elements such as buttons, menus, navigation options, form fields and many more.
    \item \textbf{SCSS:} To complete our front-end stack we utilize SCSS. It provides additional functionality like
        variables, nested operations, and it makes it possible to dynamically render themes for Angular Material.
\end{itemize}

\subsection{Backend}\label{subsec:backend}

\begin{itemize}
    \item \textbf{NodeJS:} % TODO @joernneumeyer
    \item \textbf{Express:} % TODO @joernneumeyer
    \item \textbf{MongoDB:} % TODO @joernneumeyer
    \item \textbf{MySQL:} % TODO @joernneumeyer
\end{itemize}
