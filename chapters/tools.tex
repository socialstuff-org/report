%!TEX root = ../report.tex
\chapter{Tools}\label{ch:tools}

This chapter covers all tools we used during the execution of the project.


\section{Version Control}\label{sec:version-control}

Version control, also known as source code control, is the practice of managing and tracking changes to software code.
Version control systems such as Git or Subversion provide software development teams with useful features to track
changes on software code.
They are particularly powerful in collaboration of teams were multiple engineers are working on the sames application
or even the same components of an application.
The control system our choice for this project was Git hosted by GitHub.
With GitHub, we were able to contribute to the project in parallel by utilizing several branches, so we could work in
parallel.
Further, it allowed us to create pull requests for those branches as soon as a feature or sub part of the application
has been finished.
Those pull requests were then reviewed by someone else so that we could ensure that only approved and compliant code
went into the application.

\section{Integrated development environment (IDE)}\label{sec:integrated-development-environment-(ide)}

As integrated development environment (IDE) we had no given specification so everyone utilized their IDE of choice.
The utilized IDEs were mainly Visual Studio Code and PHPStorm by Jetbrains.

% TODO @all enrich this section with more information

\section{Project Management Tools}\label{sec:project-management-tools}

For project management we utilized several tools.
For task tracking we focussed on Trello boards which provide several lists which can be named as preferred.
In our case we created a board with six different lists / columns:

\begin{itemize}
    \item TODO - tasks which are scheduled to be done in the project timeframe
    \item In Progress - tasks which are currently being worked on.
            This could be analysis or design tasks but also implementation.
    \item Review - tasks which are currently being reviewed or waiting for a review.
            This could be reviewing diagrams or code review on a pull request.
    \item Testing - tasks related to implementation which are not fully tested or where tests are not yet passing.
    \item In documentation - tasks which are functionality wise finished but still need to be documented
    \item Completed - tasks which are fully completed - no further action to be done.
\end{itemize}

We also made use of Visual Paradigm to create more complex diagrams such as a 
\textbf{\hyperref[sec:work-breakdown-structure-(wbs)]{work breakdown structure}}.

\section{Visual Design Tools}\label{sec:visual-design-tools}

For creating all sort of visual design related artefacts we utilized Adobe XD\@.
Adobe XD is a proprietary software offering a free plan to create user interface (UI) and user experience (UX)
related artefacts such as wireframes, design guidelines, grids as well as detailed designs.
